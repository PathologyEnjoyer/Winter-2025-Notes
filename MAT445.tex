
\documentclass[x11names,reqno,14pt]{extarticle}
\input{preamble}
\usepackage[document]{ragged2e}
\usepackage{epsfig}
\usepackage{dynkin-diagrams}
\usepackage{pgfkeys}
\usepackage{pgfopts}
\usepackage{xcolor}
\usepackage{ytableau}

\pagestyle{fancy}{
	\fancyhead[L]{Winter 2025}
	\fancyhead[C]{MAT445F}
	\fancyhead[R]{John White}
  
  \fancyfoot[R]{\footnotesize Page \thepage \ of \pageref{LastPage}}
	\fancyfoot[C]{}
	}
\fancypagestyle{firststyle}{
     \fancyhead[L]{}
     \fancyhead[R]{}
     \fancyhead[C]{}
     \renewcommand{\headrulewidth}{0pt}
	\fancyfoot[R]{\footnotesize Page \thepage \ of \pageref{LastPage}}
}
\newcommand{\pmat}[4]{\begin{pmatrix} #1 & #2 \\ #3 & #4 \end{pmatrix}}
\newcommand{\A}{\mathbb{A}}
\newcommand{\B}{\mathbb{B}}
\newcommand{\fin}{``\in"}
\newcommand{\mk}[1]{\mathfrak{#1}}
\newcommand{\g}{\mk{g}}
\newcommand{\h}{\mk{h}}
\newcommand{\J}{\mc{J}}
\newcommand{\tphi}{\tilde{\phi}}
\newcommand{\pois}[2]{\{#1,#2\}}
\newcommand{\fibrate}[3]{\begin{tikzcd} #1 \ar[d, "#2"] \\ #3 \end{tikzcd}}
\newcommand{\bark}{\bar{k}}
\renewcommand{\t}{\mk{t}}
\DeclareMathOperator{\Perm}{Perm}
\DeclareMathOperator{\pdim}{pdim}
\DeclareMathOperator{\gldim}{gldim}
\DeclareMathOperator{\lgldim}{lgldim}
\DeclareMathOperator{\rgldim}{rgldim}
\DeclareMathOperator{\idim}{idim}
\DeclareMathOperator{\SU}{SU}
\DeclareMathOperator{\SO}{SO}
\DeclareMathOperator{\Ad}{Ad}
\DeclareMathOperator{\ad}{ad}
\DeclareMathOperator{\gr}{gr}
\DeclareMathOperator{\Sig}{Sig}
\DeclareMathOperator{\Res}{Res}
\newcommand{\Rmod}{R-\text{mod}}
\newcommand{\RMod}{R-\text{Mod}}
\newcommand{\onto}{\twoheadrightarrow}
\newcommand{\into}{\hookrightarrow}
\newcommand{\barf}{\bar{f}}
\newcommand{\dd}[2]{\frac{d#1}{d#2}}
\newcommand{\pp}[2]{\frac{\partial #1}{\partial #2}}
\newcommand{\gl}{\mk{g}\mk{l}}
\newcommand{\spew}{\Sp(E,\omega)}
\newcommand{\jew}{\mc{J}(E,\omega)}
\renewcommand{\P}{\mathbb{P}}
\renewcommand{\E}{\mathbb{E}}
\DeclareMathOperator{\Ext}{Ext}
\DeclareMathOperator{\Rank}{Rank}
\DeclareMathOperator{\Sp}{Sp}
\DeclareMathOperator{\ann}{ann}
\DeclareMathOperator{\Lag}{Lag}
\DeclareMathOperator{\Riem}{Riem}
\DeclareMathOperator{\Span}{span}
\DeclareMathOperator{\Ind}{Ind}
\newcommand{\exactlon}[5]{
		\begin{tikzcd}
			0\ar[r]&#1\ar[r,"#2"]& #3 \ar[r,"#4"]& #5 \ar[r]&0
		\end{tikzcd}
}


\title{MAT445}
\author{John White}
\date{Winter 2025}


\begin{document}

\section*{Lecture 1 - 7/1/25}

Missed :(

\section*{Lecture 2 - 9/1/25}

Missed :(

\section*{Lecture 3 - 14/1/25}

\subsection*{\underline{Character theory}}

Consider $\dim\Hom_G(\rho_i,\rho_j) = 1$ if $i = j$ and $0$ if $i \neq j$ (meaning if $\rho_i\not\cong\rho_j$)

Recall: Given a representation $\rho:G\to \GL_n(k)$, the character of $\rho$, $\chi_\rho$, is given by $\chi_\rho:G\to k, g \mapsto \tr(\rho(g))$

For today, $G$ will be finite, $k = \bar{k}$ will be algebraically closed, of characteristic 0. 

Basic properties of characters: 
\begin{enumerate}

\item Suppose $\rho:G\to\GL_n(k)$ is a representation: then $\chi_\rho(e) = n = \dim \rho$. 

\item $\chi_\rho(g) = \chi_\rho(hgh^{-1})$ for all $g, h \in G$, i.e. $\chi_\rho$ is constant on each conjugacy class of $G$. 

\defn

A function $f:G\to k$ which is constant on conjugacy classes is called a \underline{class function}. 

The $\rho_i$ (isomorphism classes of reps) will form an ONB for the space of class functions.  

Given $\rho_1:G\to \GL_n(k), \rho_2:G\to \GL_m(k)$, $\chi_{\rho_1\oplus\rho_2} = \chi_{\rho_1} + \chi_{\rho_2}$

\item $\chi_{\rho_1\otimes\rho_2} = \chi_{\rho_1}\chi_{\rho_2}$

To see this, let $A, B$ be diagonalizable (which we have WLOG because the image of any finite group are all diagonalizable over an algebraically closed $k$ of char 0, which follows from Jordan Normal form)

Then $\tr(A\otimes B)=\tr(A)\tr(B)$. 

I can't see the board he's writing on very well, and also I am not sure how $A\otimes B$ was defined. 

\end{enumerate}


\claim

$\chi_\rho:G\to k$ always factors through $\Q(\mu_\oo)$, the subfield of $k$ containing $\Q$ ($k$ has char 0) generated by all roots of unity ($k = \bar{k}$)

\proof

Because $G$ is finite, $\rho_G$ has finite order, hence its eigenvalues are roots of unity, so the trace is the sum of roots of unity. 

\qed

\defn

$\bar{\cdot}:\Q(\mu_\oo) \to \Q(\mu_\oo)$  is the unique field homomorphism with the property that $\bar{\zeta} = \zeta^{-1}$ for all roots of unity $\zeta \in \Q(\mu_\oo)$. 

\begin{enumerate}

\item[5] $\chi_{\rho^v}= \bar{\chi_\rho}$

Recall $\rho^v$ is defined via the formula $g\cdot f = f(g^{-1}\cdot -)$ where $f$ is a functional. We have 
\begin{align*}
\chi_{\rho^v}(g) & = \tr(p(g^{-1})) \\
& = \sum_{\zeta\text{ is an eigenvalue of }\rho(G)}\zeta^{-1} \\
& = \sum \bar{\zeta} = \bar{\chi_\rho(g)}
\end{align*}

This also follows from the Hom-tensor adjunction because $\Hom_k(\rho_1,\rho_2) = \rho_1^v\otimes\rho_2$. 

\end{enumerate}

\defn

Let $\chi,\psi:G\to \Q(\mu_\oo)$ be class functions. We define their inner product by 
\[
\langle\chi,\psi\rangle = \frac{1}{|G|}\sum_{g\in G}\chi(g)\bar{\psi(g)}
\] 
This indeed is a positive definite non degenerate. 

Let $\rho_1:G\to\GL_n(k), \rho_2:G\to\GL_m(k)$. What is $\langle\chi_{\rho_1},\chi_{\rho_2}\rangle$? 
\thm
\[
\langle\chi_{\rho_1},\chi_{\rho_2}\rangle = \dim_k\Hom_G(\rho_1,\rho_2) = \dim_k\Hom(\rho_1,\rho_2)^G
\]

\cor

Suppose $\rho_1,\rho_2$ are irreducible. Then $\langle\chi_{\rho_1},\chi_{\rho_2}\rangle$ is 0 if $\rho_1,\rho_2$ are not isomorphic, and $1$ if they are. So the $\rho_i$ form an orthonormal basis for the space of class functions. 

\qed

\proof

Let $R_G \in k[G]$ be the element given by 
\[
R_G = \frac{1}{|G|}\sum_{g\in G}eg
\]

We want to show 
\begin{enumerate}

\item for $v \in V^G, R_G\cdot v = v$. 

\item For arbitrary $v \in V, R_G\cdot v \in V^G$

\end{enumerate}

To check:

\begin{enumerate}

\item We have 
\begin{align*}
R_G\cdot v & = \frac{1}{|G|}\sum_{g\in G}e_g\cdot v \\
& = \frac{1}{|G|}\sum_{g\in G}v \\
& = v\\
\end{align*}

\item Fix $g \in G$. Then 
\begin{align*}
g\cdot R_G\cdot v & = g\cdot (\frac{1}{|G|}\sum_{h\in G}hv) \\
& = \frac{1}{|G|} \sum_{h\in G}gh\cdot v \\
& = \frac{1}{|G|}\sum_{h\in G}h\cdot v \\
& = R_G\cdot v
\end{align*}
\end{enumerate}

\cor

Let $V$ be a $G$-representation. Then $\dim_kV^G = \tr(R_G|V)$

\proof

\claim

$\tr($projection$) = \dim_k\Im$

\proof

Claim$\implies$Cor follows from $\tr(R_G) = \dim\Im(R_G|V) = \dim_kV^G$

\qed

We can finally prove the theorem: 

\proof

\begin{align*}
\dim_k\Hom_G(\rho_1,\rho_2) & = \dim_k\Hom_k(\rho_1,\rho_2)^G \\
& = \tr(R_G|\Hom_k(\rho_1,\rho_2)) \\
& = \tr(\frac{1}{|G|}\sum e_g|\Hom_k(\rho_1,\rho_2)) \\
& = \frac{1}{|G|}\tr(g|\hom_k(\rho_1,\rho_2)) \\
& = \frac{1}{|G|}\sum_{g\in G}\chi_{\hom_k(\rho_1,\rho_2)}(g) \\
& = \frac{1}{|G|}\sum_{g\in G}\bar{\chi_{\rho_1}}\chi_{\rho_2} \\
& = \langle\chi_{\rho_1},\chi_{\rho_2} \rangle \\
& = \bar{\underbrace{\langle\chi_{\rho_2},\chi_{\rho_1}\rangle}_{\in \Z}} \\
\end{align*}

\qed

\section*{Lecture 4, 16/1/24}

As always, $G$ will be a finite group, $k = \bar{k}$ is an algebraically closed field of characteristic 0. 

$\Q(\mu_\oo)$ is the algebraically closed subfield of $\C$ which contains all the roots of unity, and this comes with the complex conjugate $\bar{\cdot}, \zeta\mapsto\zeta^{-1}$. 

Goal: Classify finite dimensional $G$-representations over $k$. 

We have done:
\begin{enumerate}

\item Maschke's theorem, which states that any $G$-rep in $V$ over $k$ is semisimple. 

\item Character theory: $V \sim \chi_V:G\to\Q(\mu_\oo)\subseteq k$ , $g \mapsto \tr(g|V)$

\end{enumerate}

\defn

$Cl(G)$ denotes the class functions $G \mapsto \Q(\mu_\oo)$, and it is equipped with an inner product, 
\[
\langle\psi,\varphi\rangle = \frac{1}{|G|}\sum_{g\in G}\psi(g)\bar{\varphi(g)}
\]

Remark: There is an isomorphism $Cl(G) \simeq Z(\Q(\mu_\oo)[G])$, sending $\varphi$ to $\sum_{g\in G}\phi(g)e_g$

Warning: They come with different ring structures which are not preserved by this isomorphism. 

Last time we used the Reynolds operator to show $\langle\chi_V,\chi_W\rangle = \dim_k\Hom_G(V, W)$. 

If $\rho_1,\rho_2$ are irreps of $G$, then $\langle\chi_{\rho_1},\chi_{\rho_2}\rangle$ is $1$ if $\rho_1\cong\rho_2$, and 0 otherwise. 

\cor

$\#$ of conjugacy classes of irreducible representations of $G \leq \dim_{\Q(\mu_\oo)}Cl(G) = \#$ of conjugacy classes of $G$

\proof

If $\chi_{\rho_i}$ are orthonormal, then the number of conjugacy classes of irreps is equal to $\dim \Span(\chi_{\rho_i}) \subseteq Cl(G)$, so this number is $\leq \dim Cl(G)$

\qed

\prop

Let $V$ be a $G$-representation. Then 
\[
\Phi_V:\oplus_{\rho_i\text{ irrep of }G}\rho_i\otimes_k\Hom_G(\rho_i, V) \to V
\]
given by $v\otimes f \mapsto f(v)$ is an isomorphism. 

\proof

First, we show it is surjective. By Maschke, $V = \oplus_{\rho_i \text{ reps }G}\rho_i^{n_i}$. 

Let $v \in \rho_i^{n_i}\subseteq V$, $v = (v_1, \dots, v_{n_i})$. Let $f_j:\rho_j\to \rho_i^{n_i}$ be the inclusion of the $j$th coordinate.

Then $\Phi_v(\sum_jv_j\otimes f_j) = v$. 

Now we show injectivity. 

We have
\[
\dim_k\oplus\rho_i\otimes_k\Hom_G(\rho_i,V) = \dim_kV
\]
This follows from 
\[
\dim_k\Hom_G(\rho_i,V) = n_i
\]
This follows from 
\begin{align*}
\Hom_G(\rho_i,V) & = \Hom_G(\rho_i,\oplus\rho_i^{n_i}) \\
& = \oplus_j\Hom_G(\rho_i,\rho_j)^{n_i} \\
& = \Hom_G(\rho_i,\rho_j)^{n_i} \\
\end{align*}
Which is $n_i$-dimensional
\[
\dim_k\oplus\rho_i\otimes\Hom_G(\rho_i,V) = \sum n_i\dim_k\rho_i = \dim V
\]

\qed

\cor

\[
V \simeq \oplus_{\rho\text{ irreps of }G}\rho_i^{\langle\chi_{\rho_i},\chi_{V}\rangle}
\]

\proof

Enough to show $\rho_i^{\langle\rho_i,V\rangle } \simeq \rho_i\otimes_k\Hom_G(\rho_i,V)$, i.e. $\dim_k\Hom(\rho_i,V) = \langle\chi_{\rho_i},\chi_{\rho_j}\rangle$. But that's the theorem. 

\qed

\cor

\[
V \simeq \oplus_{\rho_i\text{irreps}}\rho_i^{\oplus n_i}
\]
, then $\langle \chi_V,\chi_V\rangle = \sum_in_i^2$

\proof

$\chi_V = \sum n_i\chi_{\rho_i}$

\qed

\cor

$V \simeq W\iff \chi_V = \chi_W$

\cor

$V$ is irreducible if and only if $\langle\chi_V,\chi_V\rangle = 1$. 

\proof

Write $V = \oplus_i\rho_i^{n_i}$: so $\langle\chi_V,\chi_V\rangle = \sum_in_i^2$ is equal to 1 iff exactly $1$ $n_i$ is nonzero, and equal to 1. 

\qed

\exm(The regular representation)

Let $G \curvearrowright k(G)$ via left multiplication. 

$\chi_{k[G]}(g) = \tr(g|k[G])$, which is $|G|$ if $g$ is the identity, and 0 otherwise. 

Because $g\cdot e_{g'} = e_{gg'}$, we have
\[
\tr(g|k[G]) = \#\{h\in G \mid gh = g\}
\]

Remark: if $X$ is a $G$-set (i.e. a set with a $G$-action), then the permutation representation, $k^X,$ has character 

\[
\chi_{k^X}(g) = \#\{x\in X \mid g\cdot x = x\}
\]

\cor

As a $G$-representation, 
\[
k[G] \simeq \oplus_{\rho_i\text{ irrep }}\rho_i^{\oplus\dim\rho_i}
\]

\proof

\begin{align*}
\langle\chi_{\rho_i},\chi_{k[G]}\rangle & = \frac{1}{|G|}\sum_{g\in G}\chi_{\rho_i}(g)\bar{\chi_{k[G]}(g)} \\
& = \frac{1}{|G|}\chi_{\rho_i}(e)\bar{\chi_{k[G]}(e)} \\
& = \frac{1}{|G|}\dim\rho_i |G| \\
& = \dim\rho_i
\end{align*}
Because this representation is 0 except at the identity. 

\qed

Remark: In fact, $\Hom_G(k[G],\rho_i)\simeq\rho_i$, AS A VECTOR SPACE.

\proof

$\Hom_G(k[G],\rho_i) = \Hom_{k[G]}(k[G],\rho_i) \simeq \rho_i$ AS A VECTOR SPACE

\qed

\cor

Let $\rho_i$ be the (conjugacy classes of) irreps of $G$, $n_i$ the dimension of $\rho_i$. 

Then $\sum_in_i^2= |G|$. 

\proof

$|G| = \dim_kk[G] = \dim_k\oplus_i\rho_i^{\oplus\dim\rho_i} = \sum n_i^2$

\qed

\thm

Let $G$ be a finite group, $k = \bar{k}$ an algebraically closed field of characteristic 0, $\rho_1,\dots,\rho_n$ the irreps of $G$. Then $\{\chi_{\rho_i}\}$ is an orthonormal basis of $Cl(G)$.

\proof

We know it's orthonormal (so in particular linearly independent), so it is left to show that this indeed spans all of $Cl(G)$. 

What remains to show is that $\chi_{\rho_i}$ span $Cl(G)$.

It is enough to show that if $\psi\in Cl(G)$ with $\langle\psi,\chi_{\rho_i}\rangle = 0$ for all $i$, then $\psi = 0$, i.e. the orthogonal complement of the span of the $\chi_{\rho_i}$ is trivial. 

\defn

If $\psi:G\to\Q(\mu_\oo)$ is a class function, 
\[
\gamma_\psi\eqdef\sum_{g\in G}\psi(g)e_g \in Z(k[G])
\]

\exm

If $\psi:G\to k$, $g \mapsto \frac{1}{|G|}$, $\gamma_\psi = R_G$. 

We will compute what $\gamma_\psi$ does to a representation. 

\prop 

If $\rho$ is an irreducible representation of $G$, then $\gamma_\psi:\rho\to\rho$ is multiplication by the scalar $\frac{|G|}{\dim\rho}\langle\psi,\chi_{\rho^v}\rangle$

\proof
\,

\begin{enumerate}

\item First, $\gamma_\psi:\rho\to\rho$ is a homomorphism of $G$-representations, which follows from $\gamma_\psi\cdot g \cdot v = g\cdot \gamma_\psi\cdot v$ for all $g \in G, v \in \rho$, as $\gamma_\psi\in Z(k[G])$.

\item By Schur, $\gamma_\psi:\rho\to\rho$ is a scalar. 

\item $\gamma_\psi = \frac{\tr(\gamma_\psi|\rho)}{\dim\rho}\cdot\Id_\rho$, so
\[
\tr(\gamma_\psi|\rho) = \tr(\sum_{g\in G}\psi(g)e_g|\rho) = \sum_{g\in G}\psi(g)\chi_\rho(g) = |G|\langle\psi,\bar{\chi_\rho}\rangle = |G| \langle\psi, \chi_{\rho^v}\rangle
\]

\end{enumerate}

\qed

Now, consider $\gamma_\psi:k[G]\to k[G]$. This is zero as $\gamma_\psi$ acts as zero on every irrep (because it pairs to zero with all the irreps), and because it sends $1$ to $\gamma_\psi$, $\gamma_\psi$ has to be zero. 

\qed

\cor(of earlier claim)

$\frac{\dim\rho_i}{|G|}\gamma_{\chi_{\rho_i^v}}$ acts as $1$ on $\rho_i$, and 0 on $\rho_j$, for $\rho_i\neq\rho_j$ are irreps. 

\proof

\qed

\cor

Given any $V = \oplus\rho_i^{\oplus n_i}$, 
\[
\frac{\dim\rho_i}{|G|}\gamma_{\chi_{\rho_i^v}}
\]
acts as a projection onto $\rho_i^{n_i}\subseteq V$, which is called the $\rho_i$ isotypic part of $V$. 

\cor

$\#$irreps of $G = \#$conjugacy classes of $G$

\proof

Let $\{\rho_i\}$ be the irreps of $G$ (up to conjugacy (i.e isomorhpism)). 

Then $\{\chi_{\rho_i}\}$ is a basis for $Cl(G)$, so $\#$ of irreps = $\dim_kCl(G) = \#$conjugacy classes of $G$.

Remark: These two numbers are equal, but there is no natural or canonical bijection between the two sets in general.

\section*{\underline{Classifying rep'ns}}

\thm $G$ is abelian iff all irreps of $G$ are 1-dimensional.

\proof

Let $V$ be an irrep. If $G$ is commutative, then $\cdot g :V\to V$ is a $G$-homomorphism for all $g\in G$. 

By Schur, each $g \in G$ acts as a scalar. Now every subspace of $V$ is a subrep, hence $V$ is 1-dimensinoal. 

Now suppose that all irreps are 1-dimensional. Let $n_i$ be the dimensions of the irreps $\rho_i$, and let $c$ be the number of conjugacy classes (or equivalently the number of irreps) of $G$. Then $|G| = \sum_in_i^2$, but this is at least $c$, because we are taking the sum of $c$ positive numbers, but each $n_i$ is 1, so each element of $G$ is its own conjugacy class. 

\qed

\exm

Take $G = \Z/n\Z$

For each element $\zeta \in \mu_n \eqdef $nth roots of unity, consider $\chi_\zeta:\Z/n\Z \to k^*, a \mapsto \zeta^a$

This gives $n$ distinct reps, which is the number of conjugacy classes, hence we have a complete list. 

\exm

$S_3$ has conjugacy classes $[e], [(12)], [(123)]$, so there are 3 irreducible representations. We have a trivial representation, whose character sends all conjugacy classes to 1. 

We also have $sgn:S_3\to\{\pm1\}\subseteq k^*$, so $\chi_{sgn}$ sends $[e]$ to 1, $[(12)]$ to -1, and $[(123)]$ to 1. 

At this point we know there must be a third representation, $std$, and we can fill in its row in the character table somehow. $std$ is given by $S_3 \curvearrowright \C^{\{1,2,3\}}/\begin{pmatrix} 1 \\ 1 \\ 1 \end{pmatrix}$, with $\chi_{std} = \chi_{\C^{\{1,2,3\}}} - \chi_{triv}$, so $\chi_{st}(e) = 2, \chi_{std}(12) = 0, \chi_{std}(123) = -1$. 

We claim that $\chi_{std}$ is irreducible. To see this, we compute 
\[
\langle\chi_{std},\chi_{std}\rangle=\frac{1}{6}(2^2 + 3*0^2 + 2(-1)^2) = 1. 
\]

\exm

$Q_8 = \langle \pm1,\pm i,\pm j, \pm k\rangle$, with multiplication given as in the quaternion group, $i^2 = j^2 = k^2 = ijk = -1$. 

Conjugacy classes: $(e)$,$-1$, $\{\pm i\}$,$\{\pm j\}$,$\{\pm k\}$. 

$\chi_{triv}$ sends them all to 1, of course. 

\section*{Lecture 5, 21/1/25}

\begin{center}
\begin{tabular}{ c| c c c c c }
 & 1 & -1 & \{i, -i\} & \{j,-j\}&\{k,-k\} \\
\hline
triv & 1 & 1 & 1 & 1 & 1 \\
i-ker & 1 & 1 & 1 & -1 & -1 \\
j-ker & 1 & 1 & -1 & 1 & -1 \\
k-ker & 1 & 1 & -1 & -1 & 1 \\
$?$ & $\cdots$&$\cdots$ &$\cdots$ &$\cdots$ &$\cdots$  \\
\end{tabular}
\end{center}

Let $\mathbb{H} = \R\langle 1, i, j, k \rangle$. Then $Q_8 \curvearrowright \mathbb{H}$ by left multiplication, $\mathbb{H} \curvearrowleft \C$ by multiplication by $i$ on the right. This example might be useful to think about for the homework. 

Now let's get the character table for $S_4$. 
\begin{center}
\begin{tabular}{c | c c c c c}
\text{conj class} & 0 & (12) & (123) & (12)(134) & (1234) \\
\text{size} & 1 & 6 & 8 & 3 & 6 \\
\hline 
sgn & 1 & -1 & 1 & 1 & -1 \\
$std = \C^4/\begin{pmatrix}1\\1\\1\\1\end{pmatrix}$ & 3 & 1 & 0 & -1 & - 1 \\
$std\otimes sgn$ & 3 & -1 & 0 & -1 & 1 \\
$std\circ\pi_{4\to3}$ & $\cdots $ & $\cdots$ & $\cdots$ & $\cdots$ & $\cdots$ 
\end{tabular}
\end{center}
If $S_4$ is the symmetries of a tetrahedron, then $\pi_{4\to3}$ is the map from $S_4$ to $S_3$ furnished by $S_4$ acting on pairs of sides, of which there are 3. 

\underline{How does the structure of $G$ interact with its representation theory?}

\prop(Homework) 

Let $G, H$ be groups, (NOT necessarily finite!), $k = \bark$ algebraically closed.

Then any irrep of $G\times H$ has the form $V\boxtimes W$, where
\begin{itemize}

\item $V$ is an irrep of $G$,

\item $W$ is an irrep of $H$

\item $(g,h)\cdot v\boxtimes w = (g\cdot v)\boxtimes (h\cdot w)$

This is the same as tensoring the two reps of $G\times H$ we get from 
\[
\begin{tikzcd}
& G \ar[r] & \GL(V) \\
G\times H \ar[ur] \ar[dr] & & \\
& H \ar[r] & \GL(W) \\
\end{tikzcd}
\]

\end{itemize}

\proof

HW

\qed

We have now classified (modulo the homework) all representations of all finite abelian groups. 

In some sense, (the sense of Artin's theorem) is that the representation theory of a group is controlled by the rep theory of its abelian subgroups. 

\underline{Restriction \& induction}

Let $H \subseteq G$ be a subgroup of $H$, $G$ again finite. 

We have a restriction functor $Res_H^G: Rep_G\to Rep_H$, 
\[
(\rho:G\to\GL(W)) \mapsto \rho|_H
\]
There is a functor going the other way called induction, $Ind_H^G:Rep_H\to Rep_G$. 

\defn

Let $V$ be an $H$-representation. Then 
\[
Ind_H^GV \eqdef k[G]\otimes_{k[H]}V
\]

Equivalent descriptions: 

\[
Ind_H^G(V) \eqdef \{\phi:G\to V \mid \phi(gh^{-1}) = h\phi(g)\forall g \in G, h \in H\}
\]

An element of the former looks like $\sum_g e_g\otimes v_g$. Take $e_ge_h\otimes v = e_g\otimes (h\cdot v)$, $g\cdot \phi = g\phi(g^{-1}-)$. Think about this and see how this makes the descriptions the same. One more description: 
\[
Ind_H^G(V) = \bigoplus_{g\in G/H}g_i\cdot V
\]
where $g\cdot \sum g_iv_i = \sum g_{j(i)}k_i \cdot V $ where $g_jg_i = g_{j(i)}$ (???)

Exercise: check the above is equivalent to the other two things. 

\exm
\,
\begin{enumerate}

\item $Ind_H^G triv = k^{G/H}$ follows from second description. By definition, $Ind_H^Gtriv = \{f:G\to k\mid f(gh^{-1}) = h\cdot f(g) = f(g)\} = \{f:G/H\to k\}$

\item $Ind_{(1)}^Gk = k[G] \otimes_kk = k[G]$

\item Suppose $\chi:H\to\C^\times$ is a representation. What is $Ind_H^G\chi$? To find $Ind_H^G\chi(g)$, pick coset representative $g_i$ from $G/H$, and we get permutation matrix for $G\curvearrowright G/H$ times the diagonal matrix whose $i$th entry is $\chi(h_i)$, where $gh_i^{-1} = g_{j(i)}h_i^{-1}$
\end{enumerate}

\section*{Lecture 6, 23/1/25}

Corrections:

In the homework, problem 4 part a) should include the assumption that the action of $G$ on $H$ by conjugation is inner, i.e. for all $g \in G$, the map $(\cdot)^g:H\to H$ sending $h \mapsto g h g^{-1}$ is $(\cdot)^{h'}$ for some $h' \in H$. 

Remark: An example is if we take $G = A\times B, H = A\times\{1\}$. Then $(\cdot)^{(a,b)} = (\cdot)^{(a,1)}$

Last time:
\begin{itemize}

\item We did character tables for $Q_8,S_4$ 

\item We stated the classification of irreducible representations of a product $G\times H$

\item Classification of irreps of finite abelian groups

\item Restriction \& induction

\end{itemize}

Here is more on induction: 

$\Ind_H^G(V) \eqdef k[G]\otimes_{k[H]}V$, where $k[G]$ is a right module and $V$ is a left one. Tensoring a right with a left yields an abelian group (indeed a $k$-vector space), and it all works out because $k[G]$ is a left $k[G]$ module. 

It is also the set $\{\phi:G\to V \mid \phi(gh^{-1}) = h\cdot\phi(g)$ for all $g \in G, h \in H\}$, where 
\[
g\cdot\phi = \phi(g^{-1}\cdot)
\]

\underline{Explanation}

An element of $k[G]\otimes_{k[H]}V$ is a formal sum $\sum e_g\otimes v_g$ such that $e_ge_h\otimes v = e_g\otimes(h\cdot v)$

How to recognize induced representations:

\begin{itemize}

\item Suppose $V$ is a $G$-rep, $W \subseteq V$ is $H$-stable. When is $V \simeq \Ind_H^GW$?

\item Consider $gW\subseteq V$. Because $W$ is $H$-stable, this only depends on $[g]\in G/H$

\end{itemize}

\prop 

$V = \Ind_H^GW$ if and only if $V = \oplus_{g\in G/H}gW$

\proof

Sketch

Recall the third version, $\Ind_H^GV = \oplus_{g_i\in G/H}g_iU$

\qed

\prop

\begin{align*}
\chi_{\Ind_H^G\rho}(u) & = \frac{1}{|H|} \sum_{g\in G, g^{-1}ug \in H}\chi_\rho(g^{-1}ug) \\
& = \sum_{x\in G/H}\hat{\chi}_\rho(x^{-1}ux)\\
\end{align*}

where $\hat{\chi}_\rho(v) = \begin{cases} \chi_\rho(v) & v \in H \\ 0 & \text{ otherwise } \\ \end{cases}$

\proof

\qed

\prop 

Let $H \subseteq G$ be a subgroup of finite index. Then 
\[
\Hom_G(\Ind_H^GV,W) \simeq \Hom_H(V, \Res_G^HW)
\]

\proof

This is a special case of the tensor-hom adjunction: 
\begin{align*}
\Hom_G(\Ind_H^GV,W) & \simeq \Hom_G(k[G]\otimes_{k[H]}V, W) \\
& = \Hom_H(V,\Hom_G(k[G],W)) \\
& = \Hom_H(V, \underbrace{W}_{ \text{as an H-rep}} )\\
& = \Hom_H(V,\Res_G^HW)
\end{align*} 

\qed

\cor

Let $V$ be a representation of $H$, $W$ is a representation of $G$, both finite. Then 
\[
\langle \chi_{\Ind_H^GV},\chi_W\rangle = \langle \chi_V, \chi_{\Res_G^HW}\rangle
\]

\proof

These numbers are the dimensions of the hom-spaces, which are the same by the above. 

\qed

\thm[Artin]

Let $G$ be a finite group, $k = \bark, char k = 0$. Then the map
\[
\bigoplus_{H\subseteq G \text{cyclic}}Cl(H)\onto Cl(G)
\]
For each cyclic group $H$, it acts on characters linearly, so we can extend that to $Cl(H)$, and we can extend that to $\oplus Cl(H)$
\proof

Remark: Let $G$ be a finite group, $R(G)$ be the ``representation ring of $G$", 
\[
R(G) = \oplus_{\rho_i\text{ irreps of }G}\Z[\rho_i]
\]

with $[\rho_i]\cdot[\rho_j] = [\rho_i\otimes\rho_j]$, by writing $\rho_i\otimes\rho_j = \bigoplus_{\rho_k\text{ irreps }}\rho_k^{n_k}$

\prop

There is a map $R(G) \to Cl(G)$ sending $[\rho_i] \to \chi_{\rho_i}$. This is a ring homomorphism (because character of tensor product is pointwise product of characters).

There is an induced map $R(G) \otimes_{\Z}k \to Cl(G)$ which is an isomorphism. 

\proof

\,
\begin{enumerate}

\item These are vector spaces of the same dimension

\item The map is surjective because (for example,) characters of irreps span. 
\end{enumerate}

\qed

\cor[to Artin's theorem]

The map (linear extension of $\oplus\Ind_H^G$)
\[
\bigoplus_{H\leq G\text{ cyclic }}R(H)_k\to R(G)_k
\]
is surjective.

I.e. every representation of $G$ is a ``k-linear combo" of irreps induced from cyclic subgroups. 

\cor
\,
\begin{enumerate}

\item $\oplus_{H\leq G\text{ cyclic }} R(H)_\Q\to R(G)_\Q$ is surjective, i.e. every irreducible character of $G$ is a $\Q$-linear combination of characters induced from cyclic subgroups.

\item $\oplus_{H\leq Q \text{ cyclic }}R(H)\to R(G)$ has finite cokernel.

\end{enumerate}

\proof
\,
\begin{itemize}

\item $(1)\implies(2)$ because the image of $\Ind$ spans $R(G)$ rationally by (1), i.e. given $x \in R(G)$, there is $N$ such that $N\cdot x \in \Im(\Ind)$, so the cokernel is torsion, and torsion finitely generated abelian groups are finite. 

\item We know $(1)$ by Artin, because $\Ind_\Q\otimes_\Q k$ is surjective, as rank $r$ invariant under extension of scalars?

\end{itemize}

\qed

We now prove Artin's theorem:

\proof

It is enough to show that the adjoint map of $\oplus\Ind_H^G$ is injective. But $\langle \Ind \chi, \psi \rangle = \langle \chi, \Res \psi \rangle$, so 
\[
\bigoplus\Res_G^H :Cl(G) \to \bigoplus_{H\leq G\text{ cyclic }}Cl(H)
\]
is adjoint to $\Ind$. Now let $\psi$ be in the kernel; then $\Res_G^H\psi \equiv 0$ for all $H$, which implies $\psi \equiv 0$, so we win. 

\qed

\subsection*{\underline{Loose ends:}}

\begin{itemize}

\item Structure of $k[G]$

\item Integral theory

\item Corollary of all this discussion: if $G$ is a finite group, $\rho$ an irrep, then $\dim\rho \mid |G|$

\end{itemize}

\subsection*{\underline{Structure of $k[G]$ (and more generally, semisimple algebras)}}

\defn

Let $k$ be a field, $R$ a $k$-algebra (possibly non-commutative). Then $R$ is \underline{semisimple} if 
\,
\begin{enumerate}

\item $R$ is finite dimensional as a $k$-vector space

\item All left $R$-modules which are finite-dimensional $k$-vector spaces are semisimple. 

\end{enumerate}

\thm

Let $R$ be semisimple $k$-algebra. Then 
\[
R \simeq \prod\Mat_{n_i}(D_i)
\]
where $D_i$ are division $k$-algebras.

\proof (Take $R = k[G]$)

Conside $R$ as a left $R$-module;
\[
R \simeq \oplus M_i^{\oplus n_i}
\]
where $M_i$ is simple, all $M_j$s are mutually non-isomorphic left $R$-modules.

 Note $\Hom_{\Rmod}(M_i,M_i)$ is a division algebra (otherwise we would have a morphism with a kernel, but $M_i$ is simple). 

Because $R^{op} \simeq \Hom_{\Rmod}(R,R)$, this means 
\[
R \simeq \Hom_{\Rmod}(\oplus M_i^{\oplus n_i}, \oplus M_i^{\oplus n_i})
\]

Now, $\Hom_{\Rmod}(M_i,M_j) = 0$ for $i \neq j$ (again by simplicity and mutual nonisomorphicness) so 
\[
\Hom_{\Rmod}(R,R) \simeq \oplus_i \Hom_{\Rmod}(M_i^{n_i},M_i^{n_i})
\]

So if we take $D_i^{op} = \Mat_{n_i}(\Hom(M_i,M_i))$, we win. 

\qed

\cor

Let $k = \bark$. Then $R \simeq \oplus \Mat_{n_i}(k)$

\proof
\,
\begin{enumerate}

\item Finite dimensional central division algebras over an algebraically closed field are the field itself. 

\item Or, same proof as in Schur, 
\[
\Hom_{\Rmod}(M_i,M_i) = k
\]

\end{enumerate}

\qed

Let's specialize to $R = k[G]$. 

As a $k[G]$-module, $k[G] \simeq \rho_i^{\oplus n_i}$, so we have a map
\[
k[g]\to\bigoplus_{\rho_i\text{ irrep }} \underline{\Hom}_k(\rho_i,\rho_i) \simeq \oplus_{\rho_i\text{ irrep }}\rho_i\boxtimes\rho_i^v \simeq \oplus_{\rho_i\text{ irrep }}\rho_i\otimes\Hom(\rho_i,k[G])
\]
\[
x \mapsto \text{right multiplication by } x
\]

Recall: If $V$ is any $G$-rep, then $V = \oplus \rho_i\otimes\Hom_G(\rho_i,V)$, 

so we have $k[G] \to \oplus \End(\Hom(\rho_i, k[G]))$

\claim

This isomorphism of rings is $G\times G$-equivariant if we give $\End(\rho_i^{\dim\rho_i})$ the $G\times G$ structure $\rho_i\boxtimes\rho_i^v$

\proof

We need to check $\End(\rho_i^{\dim_i})$ as a right $G$-module it is $(\rho_i^v)^{\dim \rho_i}$.

If $G \into G\times G$ by $g\mapsto (g, g^{-1})$, then it has an invariant in $\Hom_G(\rho_i^{\dim \rho_i},\rho_i^{\dim\rho_i})$, 

As $G$-reps, $\Hom(\rho_i,\rho_i) \simeq \rho_i\otimes\rho_i^v$

\claim

Given a rep $V\boxtimes W$ of $G\times G$, the structure of $V$ and $V\boxtimes W|_{(g, g^{-1})}$ determines $W$.

\proof

\qed

\section*{Lecture 7, 28/1/25}

Substitute for today: Dr Jacob Tsimerman

Let $k = \bark$ be an algebraically closed field of characteristic $0$, $G$ a finite group. 

Let $(\rho_1,V_1), \dots, (\rho_n,V_n)$ be the irreducible left representations of $G$. 

\thm
\[
k[G] \cong \bigoplus_{i=1}^n\rho_i\boxtimes\rho_i^v = \bigoplus_{i=1}^nV_i\otimes V_i^*
\]
as $G\times G$-reps $((g,g')\cdot v\otimes v^* = (g\cdot v)\otimes v^* + v\otimes (g'\cdot v^*))$

\proof

Let $W_i \eqdef \Hom_G(V_i, k[G])$. Then 
\[
k[G] \cong \bigoplus_{i=1}^nV_i\otimes W_i
\]
as $G\times G$-representations because we get the right $G$-action for free. 

\claim

As right $G$-representations, $W_i \cong V_i^*$

\proof
\,

\underline{Convention:} Given an element $x = \sum_{g\in G}a_g(x)g \in k[G]$, we use $a_g:k[G] \to k$ to denote the $g$-th coefficient. 

This has the property that $a_g(x\cdot g') = a_{g'g^{-1}}(x)$

Define $\psi:W_i \to V_i^*$ by 
\[
\psi(\phi)\eqdef a_1\circ\phi
\]

\claim

$\psi$ is an isomorphism

\proof

Suppose $\phi\in W_i$. For $g \in G$, $a_g(\phi(v)) = a_1(g^{-1}\phi(v))$. But $\phi$ is a map of left $G$-modules, so this is $a_1(\phi(g^{-1}(v))) = \psi(\phi)(g^{-1}v)$. 

So, we can write
\[
\phi(v) = \sum_{g\in G}\psi(\phi)(g^{-1}v)\cdot v
\]
 
So $\phi$ is entirely determined by $\psi(\phi)$, or in other words, $\psi$ is injective.

On the other hand, let $\ell \in V^*$. 

Consider $\phi_\ell \in W_i$, $\phi_\ell(v) = \sum_{g\in G}\ell(g^{-1}v)\cdot g$

\claim

$\phi_\ell \in W_i$

\proof

Let $g_0 \in G$. Then 
\[
\phi_\ell(g_0v) = \sum_{g\in G}\ell(g^{-1}g_0v) = \sum_{g\in G}\ell(g^{-1}v)\cdot g_0g = g_0\cdot\phi_\ell(v)
\]

This shows that $\psi$ is surjective.

\qed

\claim

$\psi$ respects the right $G$-action. 

\proof

\begin{align*}
\psi(\phi^{g_0})(v) & = \psi(\phi)(g_0v) \\
& = a_1(\phi(g_0v))\\
& = a_1(g_0\phi(v)) \\
& = a_{g_0^{-1}}(\phi(v)) \\
\end{align*}

On the other hand, 
\begin{align*}
\psi(\phi^{g_0}v) & = a_1(\phi^{g_0}(v)) \\
& = a_1(\phi(v)g_0) \\
& = a_{g_0^{-1}}(\phi(v))
\end{align*}
So $\psi(\phi^{g_0}) = \psi(\phi)^{g_0}$
\qed

This proves the theorem. 
\qed

\underline{Matrix Coefficients}

Let $\{v_1, \dots, v_n\}$ be a basis for an irreducible representation $V$. 

Let $\{v_1^*, \dots, v_n^*\}$ be the dual basis for $V^*$.

\defn

Given $1 \leq i, j \leq m$, the \underline{matrix coefficient $a_{i,j}$} is given by 
\[
a_{i,j}(g) = v_i^*(g\cdot v_j)
\]
This is a function from $G$ to $k$. 

Define $A_{i,j} \in k[G]$ by 
\[
A_{i,j}\eqdef \sum_{g\in G}a_{i,j}(g)\cdot g 
\]

\thm
\[
\langle A_{i,j}\rangle_{1\leq i, j\leq m} = \rho\boxtimes\rho^v
\]
where $(\rho,V)$ is the $G$-rep.

\proof

\qed

\thm

Let $G$ be a finite group, $k = \bark$ an algebraically closed field of characteristic 0. 

Let $(\rho,V)$ be an irreducible representation of $G$. 

Then $\dim V \mid |G|$

\proof

\cor

If $d_1, \dots, d_n$ is the dimensions of the irreps of $G$, then 
\begin{enumerate}

\item $m = $number of conjugacy classes of $G$ (often called $m$)

\item $d_i | |G|$ for all $i$

\item $\sum_{i=1}^m d_i^2 = |G|$

\end{enumerate}

\proof

\qed

\exm

If $G = S_3$, $m = 3$, with conjugacy classes $[\Id], [(12)],[(123)]$, then we have $d_1 = 1, 1 + d_2^2 + d_3^2 = 6, d_2, d_3 \mid 6$. 

So we must have $d_2 = 1, d_3 = 2$. 

\underline{Recollections of algebraic integers}

\defn

Let $R$ be a commutative ring. 

Then $x \in R$ is \underline{integral}, or an \underline{algebraic integer}, if $x$ satisfies a monic integer polynomial. 

\exm
\,
\begin{itemize}

\item 3

\item $\sqrt{5}$

\item $\frac{1+\sqrt{5}}{2}$

\end{itemize}

Non-examples include
\begin{itemize}

\item $\frac37$

\item $\frac{1}{\sqrt{2}}$

\end{itemize}

\prop

The following are equivalent:\,
\begin{enumerate}

\item $x$ is integral

\item The subring generated by $x$ is a finitely generated $\Z$-module

\item The subring generated by $x$ is contained in a finitely generated $\Z$-module in $R$. 

\end{enumerate}

\proof

Let's start with $(1)\implies(2)$. 

Suppose $x^N + \sum_{i=1}^{N-1}a_ix^i = 0$, $a_i \in \Z$. 

Then $x^N \in \langle1,x,\dots,x^{N-1}\rangle_\Z$. But then $x^{N+1} \in \langle 1, x, \dots, x^N\rangle_\Z$, so $x^{N+1} \in \langle 1, x, \dots, x^{N-1}\rangle_\Z$.

So the subring generated by $x$ equals $\langle 1, x, \dots, x^{N-1}\rangle_\Z$.

$(2)\implies(3)$ is clear

So let's see $(3)\implies(1)$. 

Let $A_N = \langle 1, x, x^{N-1}\rangle_\Z$. By assumption, there exists a finitely generated $\Z$-module $B \subset R$ such that $A_1\subseteq A_2 \subseteq\cdots \subseteq B$ 

By Noetheriality, the sequence stabilizes, so there exists some $M$ such that $A_M = A_{M-1}$, and so $x^M$ is a finite linear combination of lower powers of $x$, so there are $a_i$ such that
\[
x^M + \sum_{i=1}^{M-1}a_ix^i = 0
\]

\cor

The things on the list of non algebraic integers actually belong on the list!

\proof

\qed

\section*{Lecture 8, 30/1/25}

Sub Prof: Mathilde Gerbelli-Gauthier

End Goal: $G$ finite, $\rho$ irrep of $G$ over $k = \bark$ algebraically closed of characteristic 0. We want to show that $\dim \rho \mid |G|$

Strategy: Prove that $\frac{|G|}{\dim\rho}$ is an algebraic integer

As a corollary of the proof of the last prop, we get

\cor

Integral elements of $R$ form a subring. 

\proof

\qed

\subsection*{\underline{Integrality of characters}}

As always, let $G$ be a finite group, $k = \bark$ algebraically closed of characteristic 0, and $\rho:G\to\GL_n(k)$ just any representation (not necessarily irreducible).

\prop
\,
\begin{enumerate}

\item The values of the character of $\rho, \chi_\rho(g)$, are algebraic integers

\item Let $u = \sum_{g\in G}u(g)g$ be an element of $Z(k[G])$. Suppose that $u(g) \in k$ are algebraic integers. Then $u$ is integral. 

\end{enumerate}

At some point in the classes I missed we show that the indicators of conjugacy classes span the center of $k[G]$. 

\proof
\,
\begin{enumerate}

\item $\chi_\rho(g)$ is a sum of roots of unity, hence a sum of algebraic integers, hence an algebraic integer.

\item Using a previous result, let $u(g)$ be the indicator function of a conjugacy class. But the sub-$\Z$-module of $Z(k[G])$ generated by the indicator functions is a subring (because the product of $1_{C_1}\cdot 1_{C_2}$ is a linear combination of the indicators of conjugacy classes, and the coefficient in front of each $g$ is an integer). 

Thus each indicator of a conjugacy class is contained in a finitely generated $\Z$-module, and is integral. 

\end{enumerate}

\qed

\cor

Let $\rho$ be an irrep of $G$ and let $u \in Z(k[G])$ be as before. Then
\[
u_\rho = \frac{1}{\dim\rho}\sum_{g\in G}u(g)\chi_\rho(g) \in k
\]
is an algebraic integer. 

\proof

\claim

Given $\rho$, $u \mapsto\frac{1}{\dim\rho}\sum u(g)\chi_\rho(g)$ is a ring homomorphism

\proof

\[
u_1 * u_2 \mapsto \left(\frac{1}{\dim\rho}\sum u_1(g)\chi_\rho(g)\right)\left(\frac{1}{\dim\rho}\sum u_2(g)\chi_\rho(g)\right)
\]

\qed

The goal will be to define a ring-hom from $Z(k[G])$ to $k$ sending $u$ to $u_\rho$. Since $u$ is integral , it maps to to an integral element of $k$. 
\[
u \mapsto \frac{|G|}{\dim\rho}\langle u, \chi_{\rho^v}\rangle = u_\rho
\]
\[
\sum u'(g)\chi_\rho(g) = |G|\langle u, \rho^v\rangle
\]

Recall that $Z(k[G]) \curvearrowright\rho$ by $G$-homomorphism, that action induces a natural map 
\[
Z(k[G]) \mapsto \Hom_G(\rho,\rho) = k
\]
So
\[
u \mapsto \frac{|G|}{\dim\rho}\langle u,\chi_{\rho^v}\rangle
\]
The matrix is scalar, so it suffices to compute its trace. Its trace is 
\[
\sum_{g\in G} u(g)\chi_{\rho}(g) = |G|\langle u,\chi_{\rho^v}\rangle
\]
Dividing by $\dim\rho$ gives the result. 

\qed ?

\thm

Let $G$ be a finite group, $k = \bark$ an algebraically closed field of characteristic 0, $V_\rho$ an irrep of $G$. Then $\dim V \mid \|G|$

\proof

Set $u = \sum_{g\in G}\chi_\rho(g^{-1})g$. By the above, we have 
\begin{align*}
\frac{1}{\dim\rho}\sum u(g)\chi_\rho(g) & = \frac{|G|}{\dim\rho}\langle\chi_{\rho^v},\chi_{\rho^v}\rangle \\
& = \frac{|G|}{\dim\rho}\underbrace{\dim\Hom_G(\rho^v,\rho^v)}_{=1} \\
& = \frac{|G|}{\dim\rho}
\end{align*}
But the left hand side is an integral element of $\Q$, so the right hand side is an integral element of $\Q$, hence an integer. 

\qed

\subsection*{\underline{Rep theory of the symmetric group}}

As always, $|G|<\oo, Char (k = \bark) = 0$

Here are some key facts about the symmetric groups: 
\begin{enumerate}

\item The number of irreps of $S_n$ is equal to the number of conjugacy classes in $S_n$. 

\item The conjugacy classes in $S_n$ (aka cycle type) are in bijection with partitions of $n$. 

\item The irreps of $S_n$ are also indexed by partitions of $n$. 

\end{enumerate}

\defn

A \underline{partition of $n$} is a sequence $\lambda = (\lambda_1 \geq \lambda_2 \geq \cdots \geq \lambda_r)$ such that $\sum\lambda_i = n$. 

\defn

The \underline{young diagram} $D_\lambda$ has $\lambda_1$ boxes in the first row, $\lambda_2$ in the second row, etc.

For example, the corresponding diagram for $\lambda = (4,3,1)$ 
\[
\ytableausetup{centertableaux}
\begin{ytableau}
\,& & & \\
& &  \\
& \\
\end{ytableau}
\]

The conjugate partition $\lambda'$  is the one such that $D_{\lambda'}$ is obtained by $D_\lambda$ by flipping along the diagonal. 

If $\lambda = (4,3,1), \lambda' = (3,2,2,1)$. Then $D_{\lambda'}$ is

\[
\ytableausetup{centertableaux}
\begin{ytableau}
\,& & \\
& \\
& \\
\\
\end{ytableau}
\]

\subsection*{\underline{Proejctions and young symmetrizers}}

An algorithm: start with $\lambda$

\begin{enumerate}

\item Number the booxes in your Young diagram $D_\lambda$ from left to right, top to bottom: you now have a young tableaux. 
\[
\ytableausetup{centertableaux}
\begin{ytableau}
1 & 2 & 3 & 4 \\
5 & 6 & 7 \\
8
\end{ytableau}
\]

\item Let $\cdot P \subseteq S_n$ be the subgroup of all permutations that preserve each row of our Young tableaux. E.g. $P \simeq S_4\times S_3 \into S_8$. 

\item $Q \subseteq S_n$ the subgroup that preserves each column of the same Young tableau e.g. $Q \simeq S_3 \times S_2 \times S_2 \into S_8$. 

In $\C[S_n]$, define $a = \sum_{p\in P}e_p, b = \sum_{q\in Q}sgn(q)e_q$

\item Suppose that $V$ is a vector space, and $S_n \curvearrowright V^{\otimes n}$ by permuting factors. 

The element $a$ symmetrizes along the rows, and projects onto 
\[
Sym^{\lambda_1}(V) \otimes \cdots \otimes Sym^{\lambda_n}(V)
\]
up to an isomorphism.

\item The element $b$ alternates along the columns and projects onto a tensor product of exterior powers indexed by $\lambda'$: 
\[
\bigwedge^{\lambda'_1}(V)\otimes \cdots \otimes \bigwedge^{\lambda'_n}(V)
\]

\item Set $c = ab$. This is called the \underline{Young Symmetrizer}

Here are some examples of Young symmetrizers:
If $\lambda = (1, \dots, 1)$, then $c$ gives the sign representation. $\lambda = (n)$ gives the trivial rep.

\end{enumerate}

\subsection*{\underline{Irreducibility and idempotency}}

\thm

A suitable nonzero scalar of $c = ab$ is an idempotent in $\C[S_n]$. Its image, when acting on the regular representation, is irreducible, and denoted $V_\lambda$.

Distinct partitions give rise to distinct (meaning nonisomorphic) representations and every irep arises from this process for a unique partition.

\cor

Every representation of $S_n$ is defined over $\Q$.

\proof

\qed

\exm
\,
\begin{itemize}

\item For $S_3$, $triv = (4), sgn = (1,1,1), std = (2,1)$ 

\item For $S_4$, $triv = (4), sgn = (1,1,1,1), std = (3, 1), std \otimes sgn = (2,1,1), S_4 \to S_3 = (2,2)$ 

\item In general, $(d,1,\cdots,1)$ corresponds to various exterior powers of the standard representation.

\end{itemize}

\thm(Hook-length formula)

Label each box $b$ in a young diagram (boxes to the right of $b$) $+$ (boxes below). 

These are called hook lengths. Then $\dim V_\lambda = \frac{n!}{\prod (\text{hook lengths of $b$})}$

\proof

\qed

\section*{Lecture 9, 4/2/25}

Let $n \in \Z_{>0}$. Our goal is to classify irreps of $S_n$. Recall:

\thm

For each partition $\lambda$ of $n$, there exists a unique isomorphism class of irrep $V_\lambda$ of $S_n$, constructed as follows: 
\[
\ytableausetup{centertableaux}
\begin{ytableau}
\,& & & & & \lambda_1 \\
& & & \lambda_2 \\
& & \lambda_3 \\
& \lambda_4 \\
\end{ytableau}
\]
where $\sum\lambda_i = n$
We let $R$ be the subgroup of $S_n$ which preserves the rows, $Q$ the subgroup preserving the columns. We set
\[
a\eqdef \sum_{g\in P}e_g \in \C[S_n]
\]
\[
b\eqdef \sum_{g\in Q}sgn(g)e_g \in \C[S_n]
\]
\[
c = ab
\]
Then $V_\lambda \eqdef \C[S_n]c$ is an irrep of $S_n$. 

Further, every irrep arises in this way. 

\proof

Summary: \, WTS
\begin{enumerate}

\item $\dim\Hom_G(V_\lambda,V_\mu) = \delta_{\mu\lambda}$

\item Any irrep is some $V_\lambda$.

\end{enumerate}

\underline{Remark:}
\,
\begin{enumerate}

\item There is an explicit dimension formula, the hook-length formula

\item There is an explicit formula for the character of $V_\lambda$ due to Frobenius. 

\end{enumerate}

For more, look for Etingof's ``Representation theory" notes for a course given at MIT. 

We will begin the proof by writing down $c_\lambda$. 

\lem

\[
c_\lambda = \sum_{g = \underbrace{p}_{\in P_\lambda}\underbrace{q}_{\in Q_\lambda}}sgn(q)e_{pq}
\]
\proof
\begin{align*}
a_\lambda b_\lambda & = \left(\sum_{g\in P_\lambda}e_g\right) \cdot \left(\sum_{h\in Q_\lambda}sgn(h)e_h\right)\\
& = \sum_{g\in P_\lambda, h \in Q_\lambda} sgn(h)\underbrace{e_ge_h}_{e_{gh}}
\end{align*}

\qed

Goal: Compute $c_\lambda^2 = a_\lambda b_\lambda a_\lambda b_\lambda$

\lem

For all $x \in \C[S_n]$, $a_\lambda x b_\lambda = \ell_\lambda(x)c_\lambda$, where $\ell_\lambda:\C[S_n]\to\C$ is some linear map. 

\cor

$c_\lambda^2 = \ell_\lambda(b_\lambda a_\lambda)c_\lambda$

\proof

Check this on each $e_g \in \C[S_n]$, $g \in S_n$. 
\begin{enumerate}

\item[Case 1] $g \in P_\lambda Q_\lambda$

We have $g = pq, e_g = e_pe_q$. 
\begin{align*}
a_\lambda e_g b_\lambda & = \left(\sum_{h \in P_\lambda}e_h\right) e_g \left(\sum_{u\in Q_\lambda}sgn(u)e_u\right) \\
& = \underbrace{\left(\sum_{h\in P_\lambda}e_he_p\right)}_{a_\lambda}\underbrace{\sum_{u\in Q_\lambda}sgn(u)e_qe_u}_{sgn(q)b_\lambda} \\
& = sgn(q)c_\lambda b_\lambda \\
& = sgn(q)c_\lambda
\end{align*}

\item[Case 2] $g \not\in P_\lambda Q_\lambda$

In this case, $a_\lambda e_g b_\lambda = 0$. To see this, it is enough to show that there exists a transposition $t \in P_\lambda$ such that $g^{-1}tg \in Q_\lambda$, i.e.  $g$ sends two elements of $\{1, \dots, n\}$ in the same row of the Young diagram for $\lambda$, to two elements of the same column. 

It is enough to show this because
\begin{align*}
a_\lambda gb_\lambda & = a_\lambda t g b_\lambda \\
& = a_\lambda g\overbrace{(g^{-1}tg)}^{sgn = -1}b_\lambda \\
& = -a_\lambda g b_\lambda \\
\end{align*}
This implies $a_\lambda gb_\lambda = 0$.

Now, suppose there do not exist 2 elements in the same row of $\lambda$ sent to the same column of $\lambda$ by $g$. 

Then $g \in P_\lambda Q_\lambda$.

To see this, let $T$ be the \underline{standard} Young Tableau for $\lambda,T' = gT$, $P'$ the stabilizer of rows of $T'$, $Q'$ the stabilizers of columns. 

\begin{enumerate}[label=(\roman*)]

\item By assumption, any two numbers in the first row of $T$ lie in different columns of $T'$.

\item Then there exists $q_1' \in Q'$ such that $q'T'$ has the same elements in first row (perhaps in a different order). 

\item Choose $p_1' \in P_\lambda$ such that $p_1'q_1'T'$ has the first row as $T$.

\item Likewise with the 2nd row and so on. 

\end{enumerate}


\end{enumerate}

\qed

\cor

\[
\ell_\lambda(b_\lambda a_\lambda) = \frac{n!}{\dim V_\lambda}
\]

\proof

later

\qed

\section*{Lecture 10, 6/2/25}

Note: For the finite group stuff we are using ``Linear reps of finite groups" by Serre (first 3rd is for chemists apparently which is amusing). Specifically chapters 1-3, 6, 9

Other stuff is also on the quercus. 

To finish the proof of the theorem, we have to show that the $V_\lambda$ are irreducible and mutually non-isomorphic. Then, from a bijection between conjugacy classes and partitions, we will be done. 

Last time we showed that $a_\lambda x b_\lambda = \ell_\lambda(x)c_\lambda$, and its corrolary, that $c_\lambda^2 = \ell_\lambda(b_\lambda a_\lambda)c_\lambda$

\cor

\[
\ell_\lambda(b_\lambda a_\lambda) = \frac{n!}{\dim V_\lambda}
\]

\proof

We know that $c_\lambda = \alpha \cdot p_\lambda$, where $p_\lambda$ is an idempotent. 
\begin{align*}
c_\lambda^2 & = \alpha^2p_\lambda^2 \\
& = \alpha^2 p_\lambda \\&= \alpha c_\lambda \\
\end{align*}
So $\alpha = \ell_\lambda(b_\lambda a_\lambda)$ so we calculate the trace of $c_\lambda:$
Trace of an idempotent is dim of its image, and $c_\lambda$ has the same image as $p_\lambda$
\begin{align*}
\tr(c_\lambda) & = \alpha\cdot\dim\Im(c_\lambda) \\
& = \ell(b_\lambda a_\lambda)\cdot\dim\Im(c_\lambda) \\
& = \ell(b_\lambda a_\lambda)\cdot \dim V_\lambda 
\end{align*}
Now, if this number is not zero, then we get an idempotent by dividing $c_\lambda$ by this number. We calculate
\begin{align*}
\tr(c_\lambda) & = \sum_{pq \in P_\lambda Q_\lambda} \tr(\cdot e_{pg})sgn(q) \\
& = \tr(\cdot\Id) \\
& = n!
\end{align*}

\qed

Goal: Compute $\dim_\C\Hom_{S_n}(V_\lambda, V_\mu) = \begin{cases} 1 & \lambda = \mu \\ 0 & \text{ otherwise } \\ \end{cases}$

We know $\Hom_{S_n}(V_\lambda,V_\mu)  = \Hom_{S_n}(\C[S_n]c_\lambda, \C[S_n]c_\mu)$

\prop

Let $A$ be a $\C$-algebra, $e \in A$ an idempotent, $M$ an $A$-module.

Then $\Hom_A(Ae,M) \simeq eM$ naturally. 

\proof

For $x \in eM$, we have a morphism $x \mapsto (a \mapsto ax)$, and $f \mapsto f(e)$. 

$e$ is an idempotent, so $1 - e$ is also an idempotent, so $1 = e + (1 - e)$, so $A \simeq Ae\oplus A(1-e)$, so $\Hom(Ae, M) \simeq \Hom(A/A(1-e),M)= \{f:A \to M \mid f(e) = f(1)\}=\{x\in M \mid x \in eM\} = eM$

\qed

Now let's prove the main theorem. 

\prop

\[
\dim_C\Hom_{S_n}(V_\lambda,V_\lambda) = 1
\]
Thus, $V_\lambda$ is irreducible

\proof

\begin{align*}
\Hom_{S_n}(V_\lambda,V_\lambda)&=c_\lambda \C[S_n]c_\lambda \\
& \subseteq a_\lambda \C[S_n]b_\lambda \\
& \subseteq \Span_{\C}(c_\lambda) \\
\end{align*}
So the dimension is at most 1. To see it is exactly 1, this space has $c_\lambda\cdot1\cdot c_\lambda \neq 0$

So $\dim = 1$, so $V_\lambda$ is irreducible.

\qed

Now let $\lambda,\mu$ be two partitions of $n$. Sat $\lambda > \mu$ if the first $\lambda_i \neq \mu_i$ has $\lambda_i > \mu_i$, i.e. the lexicographical ordering. This is a total ordering, i.e. for any pair $(\lambda,\mu)$, exactly one of $\lambda = \mu, \lambda > \mu, \lambda < \mu$ is true.

\prop

If $\lambda > \mu$, then $a_\lambda \C[S_n]b_\mu = 0$. 

\proof

In a bit

\qed

Assuming this, then, if $\lambda\neq\mu$, we want to show that $\dim\Hom_{S_n}(V_\lambda,V_\mu) = 0$.

\proof

We have 
\begin{align*}
\Hom_{S_n}(V_\lambda,V_\lambda) & = c_\lambda\C[S_n]c_\mu \\
& = a_\lambda b_\lambda \C[S_n]a_\mu b_\mu \\
& \subseteq a_\lambda \C[S_n]b_\mu \\
& = 0
\end{align*}
if $\lambda > \mu$. But $\dim\Hom_{S_n}(V_\lambda,V_\mu) = \dim\Hom_{S_n}(V_\mu,V_\lambda)$, so one, hence both, are 0. 

\qed

Now we prove the proposition

\proof

We will verify it on $e_g \in \C[S_n]$. 

\claim

There exist two numbers on the same row of the standard Young tableaux for $\lambda$, same column for $g\cdot($standard Young tableaux of $\mu$)

\proof

Homework

\qed

\exm

If $g = \Id, \lambda_1 > \mu_1$, 
\[
\ytableausetup{centertableaux}
\begin{ytableau}
\,1 & 2 & 3 & 4 \\
&  & \\
&  \\
\end{ytableau},
\ytableausetup{centertableaux}
\begin{ytableau}
\,1 & 2 & 3 \\
4 & 5 \\
& \\
\end{ytableau}
\]

Let $t$ be the transpotion for these two numbers. Then 
\begin{align*}
a_\lambda gb_\lambda & = c_\lambda t g b_\mu \\
& = a_\lambda gg^{-1}tgb_\lambda \\
& = -a_\lambda gb_\mu \\
\end{align*}

\subsection*{\underline{The rep theory of $\GL_2(\F_p)$}}

\underline{Goal:} Understand the irreps of $\GL_n(\F_q)$

What is the size of this group?

$|\GL_2(\F_q)| = (q^2 - 1)(q^2 - q) = q(q^2 - 1)(q - 1)$

\proof

\[
\GL_2(\F_q) = \{(v, w) \mid v, w \in (\F_q)^2\text{ linearly independent }\}
\]
So we can pick any $v$ a nonzero vector, and any $w$ not in the span of $v$. The number of such possible choices is $(q^2 - 1)(q^2 - q)$

\qed

\underline{Conjugacy classes:}

What are the conjugacy classes of $\GL_2(\F_q)$?

\begin{center}
\begin{tabular}{c | c | c}
Conjugacy class & number of such conjugacy classes & size of each \\
$\pmat{x}{0}{0}{x}, x \in \F_q^\times$ & $q- 1$ & 1 \\
$\pmat{x}{0}{0}{y}, x \neq y\in \F_q^\times$ & $\frac{(q-1)(q-2)}{2}$ & $q(q+1)$\\
$\pmat{x}{1}{0}{x}, x \in \F_q^\times$ & $q-1$ & $q^2 - 1$ \\
$\pmat{x}{\epsilon y}{y}{x},\epsilon$ a generator of $\F_q^\times$ & $\frac{q(q-1)}{2}$ & $q^2 - q$ \\
\end{tabular}
\end{center}
For the last one, $char\neq2$

What are the reps of $\GL_2(\F_q)$ over $\C$? Besides the trivial one, we also have $P^1(\F_1) = \{1-dim$ subspaces of $\F_q^2\}$.

This gives a permutation representation $\C^{P^1(\F_q)}$.

We have $std = \C^{P^1(\F_q)}/\C$ has dimension $q$. Let's compute the character of this representation. Let's call the first set of conjugacy classes above $z_x$, the second $d_{x, y}$, $u_x$, $t_{x,y}$

\begin{center}
\begin{tabular}{c| c c c c}
\, &$z_x$ & $d_{x, y}$ & $u_x$ & $t_{x,y}$ \\
\hline
triv & 1 & 1 & 1 & 1 \\
std & $q$ & 1 & 0 & -1 \\
\end{tabular}
\end{center}
We have 
\begin{align*}
\langle std, std \rangle & = \frac{1}{q(q-1)^2(q+1)}\left((q-1)q^2 + \frac{q(q-1)(q-2)(q+1)}{2} + 0 + \frac{q^2(q-1)^2}{2}\right) \\
& = 1
\end{align*}

What other representation are there?

Choose $\chi:\F_q^\times\to\C$, and then $(\chi\circ\det)^n$, for $n = 1, \dots, q-2$.

To construct more reps, we will examine some induces reps. 

\defn

Let $B = \pmat{*}{*}{0}{*} \subseteq \GL_2(\F_q)$ ($B$ is for Borel)

$|B| = q(q-1)^2$. Let $U$ be all the matrices of the form $\pmat{1}{*}{0}{1}$. 

What is $B/U$? It is $\F_q^\times \times \F_q^\times$. We will take reps of this and view them as reps of $B$ via the quotient map and induced reps. 

For each $\psi:\F_q^\times\times\F_q^\times\to\C^\times$, we can consider the induction $\Ind_B^{\GL_2(\F_1)}(\psi|_B)$. These are indexed by $\psi(\epsilon,1)$ and $\psi(1,\epsilon)$. 

Then $\Ind_B^{\GL_2(\F_q)}(\psi_{a,b}|_B)$ has dimension $q+1$ and has character $(q+1)\psi(x)^2$

For $d_{x, y}$ we have $\psi(x,1)+\psi(1,x) + \psi(y,1) \cdot \psi(1,y)$

I have kind of lost the plot at this point I'm sorry.

\prop

Let $\chi = \sum n_i \rho_i \in R(G)$ be a virtual character of a finite group $G$. Then $\chi$ is the character of an honest irrep iff $\langle\chi,\chi\rangle = 1$, and $\chi(1)>0$. 

\proof

If we write $\chi = \sum n_i' \rho_i'$, where $\rho_i'$ are irreps, then $\langle\chi,\chi\rangle = \sum_i(n_i)^2 = 1$ by assumption. So at most one of the $n_i$ are nonzero, and it must be $\pm 1$. So $\chi = \pm\rho$ for some irrep $\rho$. If $\chi(1)>0$, then $\chi(1)= \pm\dim\rho>0$



	



















\end{document}
