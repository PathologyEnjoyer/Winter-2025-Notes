
\documentclass[x11names,reqno,14pt]{extarticle}
\input{preamble}
\usepackage[document]{ragged2e}
\usepackage{epsfig}
\usepackage{dynkin-diagrams}

\pagestyle{fancy}{
	\fancyhead[L]{Fall 2024}
	\fancyhead[C]{MAT1344F}
	\fancyhead[R]{John White}
  
  \fancyfoot[R]{\footnotesize Page \thepage \ of \pageref{LastPage}}
	\fancyfoot[C]{}
	}
\fancypagestyle{firststyle}{
     \fancyhead[L]{}
     \fancyhead[R]{}
     \fancyhead[C]{}
     \renewcommand{\headrulewidth}{0pt}
	\fancyfoot[R]{\footnotesize Page \thepage \ of \pageref{LastPage}}
}
\newcommand{\pmat}[4]{\begin{pmatrix} #1 & #2 \\ #3 & #4 \end{pmatrix}}
\newcommand{\A}{\mathbb{A}}
\newcommand{\B}{\mathbb{B}}
\newcommand{\fin}{``\in"}
\newcommand{\mk}[1]{\mathfrak{#1}}
\newcommand{\g}{\mk{g}}
\newcommand{\h}{\mk{h}}
\newcommand{\J}{\mc{J}}
\newcommand{\tphi}{\tilde{\phi}}
\newcommand{\pois}[2]{\{#1,#2\}}
\newcommand{\fibrate}[3]{\begin{tikzcd} #1 \ar[d, "#2"] \\ #3 \end{tikzcd}}
\renewcommand{\t}{\mk{t}}
\DeclareMathOperator{\Perm}{Perm}
\DeclareMathOperator{\pdim}{pdim}
\DeclareMathOperator{\gldim}{gldim}
\DeclareMathOperator{\lgldim}{lgldim}
\DeclareMathOperator{\rgldim}{rgldim}
\DeclareMathOperator{\idim}{idim}
\DeclareMathOperator{\SU}{SU}
\DeclareMathOperator{\SO}{SO}
\DeclareMathOperator{\Ad}{Ad}
\DeclareMathOperator{\ad}{ad}
\DeclareMathOperator{\gr}{gr}
\DeclareMathOperator{\Sig}{Sig}
\newcommand{\Rmod}{R-\text{mod}}
\newcommand{\RMod}{R-\text{Mod}}
\newcommand{\onto}{\twoheadrightarrow}
\newcommand{\into}{\hookrightarrow}
\newcommand{\barf}{\bar{f}}
\newcommand{\dd}[2]{\frac{d#1}{d#2}}
\newcommand{\pp}[2]{\frac{\partial #1}{\partial #2}}
\newcommand{\gl}{\mk{g}\mk{l}}
\newcommand{\spew}{\Sp(E,\omega)}
\newcommand{\jew}{\mc{J}(E,\omega)}
\renewcommand{\P}{\mathbb{P}}
\renewcommand{\E}{\mathbb{E}}
\DeclareMathOperator{\Ext}{Ext}
\DeclareMathOperator{\Rank}{Rank}
\DeclareMathOperator{\Sp}{Sp}
\DeclareMathOperator{\ann}{ann}
\DeclareMathOperator{\Lag}{Lag}
\DeclareMathOperator{\Riem}{Riem}
\DeclareMathOperator{\Span}{span}
\newcommand{\exactlon}[5]{
		\begin{tikzcd}
			0\ar[r]&#1\ar[r,"#2"]& #3 \ar[r,"#4"]& #5 \ar[r]&0
		\end{tikzcd}
}

\title{MAT445}
\author{John White}
\date{Winter 2025}


\begin{document}

\section*{Lecture 1 - 7/1/25}

Missed :(

\section*{Lecture 2 - 9/1/25}

Missed :(

\section*{Lecture 3 - 14/1/25}

\subsection*{\underline{Character theory}}

Consider $\dim\Hom_G(\rho_i,\rho_j) = 1$ if $i = j$ and $0$ if $i \neq j$ (meaning if $\rho_i\not\cong\rho_j$)

Recall: Given a representation $\rho:G\to \GL_n(k)$, the character of $\rho$, $\chi_\rho$, is given by $\chi_\rho:G\to k, g \mapsto \tr(\rho(g))$

For today, $G$ will be finite, $k = \bar{k}$ will be algebraically closed, of characteristic 0. 

Basic properties of characters: 
\begin{enumerate}

\item Suppose $\rho:G\to\GL_n(k)$ is a representation: then $\chi_\rho(e) = n = \dim \rho$. 

\item $\chi_\rho(g) = \chi_\rho(hgh^{-1})$ for all $g, h \in G$, i.e. $\chi_\rho$ is constant on each conjugacy class of $G$. 

\defn

A function $f:G\to k$ which is constant on conjugacy classes is called a \underline{class function}. 

The $\rho_i$ (isomorphism classes of reps) will form an ONB for the space of class functions.  

Given $\rho_1:G\to \GL_n(k), \rho_2:G\to \GL_m(k)$, $\chi_{\rho_1\oplus\rho_2} = \chi_{\rho_1} + \chi_{\rho_2}$

\item $\chi_{\rho_1\otimes\rho_2} = \chi_{\rho_1}\chi_{\rho_2}$

To see this, let $A, B$ be diagonalizable (which we have WLOG because the image of any finite group are all diagonalizable over an algebraically closed $k$ of char 0, which follows from Jordan Normal form)

Then $\tr(A\otimes B)=\tr(A)\tr(B)$. 

I can't see the board he's writing on very well, and also I am not sure how $A\otimes B$ was defined. 

\end{enumerate}


\claim

$\chi_\rho:G\to k$ always factors through $\Q(\mu_\oo)$, the subfield of $k$ containing $\Q$ ($k$ has char 0) generated by all roots of unity ($k = \bar{k}$)

\proof

Because $G$ is finite, $\rho_G$ has finite order, hence its eigenvalues are roots of unity, so the trace is the sum of roots of unity. 

\qed

\defn

$\bar{\cdot}:\Q(\mu_\oo) \to \Q(\mu_\oo)$  is the unique field homomorphism with the property that $\bar{\zeta} = \zeta^{-1}$ for all roots of unity $\zeta \in \Q(\mu_\oo)$. 

\begin{enumerate}

\item[5] $\chi_{\rho^v}= \bar{\chi_\rho}$

Recall $\rho^v$ is defined via the formula $g\cdot f = f(g^{-1}\cdot -)$ where $f$ is a functional. We have 
\begin{align*}
\chi_{\rho^v}(g) & = \tr(p(g^{-1})) \\
& = \sum_{\zeta\text{ is an eigenvalue of }\rho(G)}\zeta^{-1} \\
& = \sum \bar{\zeta} = \bar{\chi_\rho(g)}
\end{align*}

This also follows from the Hom-tensor adjunction because $\Hom_k(\rho_1,\rho_2) = \rho_1^v\otimes\rho_2$. 

\end{enumerate}

\defn

Let $\chi,\psi:G\to \Q(\mu_\oo)$ be class functions. We define their inner product by 
\[
\langle\chi,\psi\rangle = \frac{1}{|G|}\sum_{g\in G}\chi(g)\bar{\psi(g)}
\] 
This indeed is a positive definite non degenerate. 

Let $\rho_1:G\to\GL_n(k), \rho_2:G\to\GL_m(k)$. What is $\langle\chi_{\rho_1},\chi_{\rho_2}\rangle$? 
\thm
\[
\langle\chi_{\rho_1},\chi_{\rho_2}\rangle = \dim_k\Hom_G(\rho_1,\rho_2) = \dim_k\Hom(\rho_1,\rho_2)^G
\]

\cor

Suppose $\rho_1,\rho_2$ are irreducible. Then $\langle\chi_{\rho_1},\chi_{\rho_2}\rangle$ is 0 if $\rho_1,\rho_2$ are not isomorphic, and $1$ if they are. So the $\rho_i$ form an orthonormal basis for the space of class functions. 

\qed

\proof

Let $R_G \in k[G]$ be the element given by 
\[
R_G = \frac{1}{|G|}\sum_{g\in G}eg
\]

We want to show 
\begin{enumerate}

\item for $v \in V^G, R_G\cdot v = v$. 

\item For arbitrary $v \in V, R_G\cdot v \in V^G$

\end{enumerate}

To check:

\begin{enumerate}

\item We have 
\begin{align*}
R_G\cdot v & = \frac{1}{|G|}\sum_{g\in G}e_g\cdot v \\
& = \frac{1}{|G|}\sum_{g\in G}v \\
& = v\\
\end{align*}

\item Fix $g \in G$. Then 
\begin{align*}
g\cdot R_G\cdot v & = g\cdot (\frac{1}{|G|}\sum_{h\in G}hv) \\
& = \frac{1}{|G|} \sum_{h\in G}gh\cdot v \\
& = \frac{1}{|G|}\sum_{h\in G}h\cdot v \\
& = R_G\cdot v
\end{align*}
\end{enumerate}

\cor

Let $V$ be a $G$-representation. Then $\dim_kV^G = \tr(R_G|V)$

\proof

\claim

$\tr($projection$) = \dim_k\Im$

\proof

Claim$\implies$Cor follows from $\tr(R_G) = \dim\Im(R_G|V) = \dim_kV^G$

\qed

We can finally prove the theorem: 

\proof

\begin{align*}
\dim_k\Hom_G(\rho_1,\rho_2) & = \dim_k\Hom_k(\rho_1,\rho_2)^G \\
& = \tr(R_G|\Hom_k(\rho_1,\rho_2)) \\
& = \tr(\frac{1}{|G|}\sum e_g|\Hom_k(\rho_1,\rho_2)) \\
& = \frac{1}{|G|}\tr(g|\hom_k(\rho_1,\rho_2)) \\
& = \frac{1}{|G|}\sum_{g\in G}\chi_{\hom_k(\rho_1,\rho_2)}(g) \\
& = \frac{1}{|G|}\sum_{g\in G}\bar{\chi_{\rho_1}}\chi_{\rho_2} \\
& = \langle\chi_{\rho_1},\chi_{\rho_2} \rangle \\
& = \bar{\underbrace{\langle\chi_{\rho_2},\chi_{\rho_1}\rangle}_{\in \Z}} \\
\end{align*}

\qed












\end{document}
